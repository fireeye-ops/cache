% Author: liliff
% created: 2011.04.08
% modified: 2013.02.27

\documentclass{article}
% packages that come with TeXLive
\usepackage[letterpaper,margin=1in,lmargin=2in]{geometry}
\usepackage[sf,tiny,compact]{titlesec}
\usepackage{ifthen}
\usepackage{hyperref}

\newcommand{\name}	{Musee Ullah}
\newcommand{\phone}	{+1 408 686 9736}
\newcommand{\email}	{milkteafuzz@gmail.com}
\newcommand{\web}	{\url{milkteafuzz.com}}
\newcommand{\location}	{Chicago, IL 60608}

\pagestyle{empty} % remove header/footer content (page numbers)
\setlength{\parindent}{0cm}
\setlength{\parskip}{0cm} % vertical space between paragraphs
\setcounter{secnumdepth}{0} % don't number sections

\newcommand{\sectionstyle}{\bfseries} % define font of sections 
\titleformat{\section}[leftmargin]{\titlerule*[0.2em]{\bfseries\_}%
  \vspace{4pt}\filleft}{\thesection}{5em}{\sectionstyle}
% The X can be replaced with any other character or string. It
% designs the "separator" before sections. Play with it.
% The characters | . _ - and * seem most desirable.
\titlespacing{\section}{0.8in}{1.5ex plus .1ex minus .2ex}{0.2in}
% First argument, width of the text space; second, vertical space
% above text; third, space between text and resume content

% this makes the title. various formatting stuff involved.
\newcommand{\makeheader}{%
  {\Large\bf\name}\hfill{}{\textsc{\location}\par}% name and location
  {\setlength{\baselineskip}{0in}\rule[-0.5mm]{\textwidth}{0.5pt}\par}%
  {\hspace{\stretch{3}}%
    \textsl{web:} \web\hspace{\stretch{1}}%
    \textsl{email:} \email\hspace{\stretch{1}}%
    \textsl{phone:} \phone\hspace{0.5em}%
  }%
  \vspace{0.5em}%
}

% 1 2  <-Use this to define your schools, companies, and organisations.
\newenvironment{entity}[2]{%
    \textbf{#1}\hfill #2\par
}{
    \vspace{0.1em}%
    \par%
}
\newenvironment{position}[2]{%
    \ifthenelse{\not\equal{#1}{}\or\not\equal{#2}{}}{\textsl{#1}\hfill#2\par}
}{
    \vspace{0.3em}%
}

\begin{document}
\makeheader
\section{Objective}

An entry-level position in systems administration or software development.
\section{Work Experience}

    \begin{entity}{\href{http://hostgator.com}{HostGator.Com, LLC}}{Austin, TX}

        \begin{position}{Systems Administrator}{June 2011 to July 2012}

-- Resolved hundreds of web hosting-related issues per week through a support ticket system.\par
-- Filed bug reports and wrote wiki documentation internally.

        \end{position}
        \begin{position}{Systems Monitoring}{December 2011 to July 2012}

-- Proactively monitored $>$4500 servers, using Zabbix, in a small team for abnormal load, user abuse, hardware health, and network connectivity/attacks.\par
-- Wrote \href{http://forums.hostgator.com/search.php?do=finduser&u=126179}{public announcements} on our \href{http://forums.hostgator.com/network-status-f14.html}{network status forums} for extended downtime issues.

        \end{position}

    \end{entity}
\section{Volunteer Experience}

    \begin{entity}{\href{http://commiesubs.com}{Commie Subs}}{Remote}

        \begin{position}{Developer}{June 2012 -- present}

-- Wrote a showtime and volunteer tracking application using the \href{http://slimframework.com}{Slim Framework}.\par
-- I write scripts and applications as needed to help everyone's productivity.

        \end{position}

    \end{entity}
    \begin{entity}{\href{http://knightsofreason.net}{Knights of Reason}}{Remote}

        \begin{position}{Moderator/System Administrator}{March 2008 -- present}

-- I moderate(d) the KoRx game servers and KoR forums as well as co-manage the server.

        \end{position}

    \end{entity}
    \begin{entity}{\href{http://thebikeproject.org}{The Bike Project of Urbana-Champaign}}{Urbana, IL}

        \begin{position}{Bike Mechanic/Technologist}{August 2010 to February 2011}

-- Assisted visitors with bicycle repair/building, repurposed old laptops for use at the shops, and managed membership information.

        \end{position}

    \end{entity}
\section{Projects}

    \begin{entity}{\href{https://github.com/liliff/zmonitor}{ZMonitor}}{}

Console client for the Zabbix monitoring suite, developed in Ruby. It interfaces with the Zabbix API using JSON for gathering current active triggers and acknowledging events. It basically provides a CLI dashboard, a method to easily acknowledge several related alerts, and one to feed output to other applications.

    \end{entity}
\section{Skills}

\textbf{OS/Distros:} CentOS, Arch Linux, Gentoo Linux, Debian, Windows XP\par
\textbf{Code:} Bash, PHP, Ruby, Javascript\par
\textbf{Markup:} Markdown, HTML5, CSS3, YAML, LaTeX, JSON\par
\textbf{HTTP:} Lighttpd, Apache, nginx\par
\textbf{Database:} MySQL, Redis\par
\textbf{Monitoring:} Zabbix, sysstat, IPMI, tcpdump/ngrep, iptables\par
\textbf{Package Managers:} portage, pacman, yum/rpm, apt\par
\textbf{Miscellaneous:} cPanel/WHM, Git, LAMP, Compilation, Jekyll\par

\section{Education}

    \begin{entity}{University of Illinois at Urbana-Champaign}{2010 -- 2011}

Completed a semester as an East Asian Languages major, with a 4.0 GPA in Japanese.
    \end{entity}

\end{document}
