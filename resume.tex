% Author: liliff
% created: 2011.04.08
% modified: 2013.03.30

\documentclass{article}
% packages that come with TeXLive
\usepackage[letterpaper,margin=.75in,lmargin=1.5in,rmargin=.75in]{geometry}
\usepackage[sf,tiny,compact]{titlesec}
\usepackage{ifthen}
\usepackage{relsize}
\usepackage[pdfborderstyle={/S/U/W 0.5},allbordercolors={0.8 0 0}]{hyperref}
\urlstyle{same}
\renewcommand{\familydefault}{\sfdefault}

\newcommand{\name}	{Musee Ullah}
\newcommand{\phone}	{1 408 686 9736}
\newcommand{\email}	{milkteafuzz@gmail.com}
\newcommand{\web}	{\url{milkteafuzz.com}}
\newcommand{\location}	{Chicago, IL 60608}

\pagestyle{empty} % remove header/footer content (page numbers)
\setlength{\parindent}{0cm}
\setlength{\parskip}{0cm} % vertical space between paragraphs
\setcounter{secnumdepth}{0} % don't number sections

\newcommand{\sectionstyle}{\bfseries} % define font of sections 
\titleformat{\section}[leftmargin]{\titlerule*[0.2em]{\bfseries\_}%
  \vspace{4pt}\filleft}{\thesection}{5em}{\sectionstyle}
% The X can be replaced with any other character or string. It
% designs the "separator" before sections. Play with it.
% The characters | . _ - and * seem most desirable.
\titlespacing{\section}{0.8in}{1.5ex plus .1ex minus .2ex}{0.2in}
% First argument, width of the text space; second, vertical space
% above text; third, space between text and resume content

% this makes the title. various formatting stuff involved.
\newcommand{\makeheader}{%
  {\Large\bf\name}\hfill{}{\location\par}% name and location
  {\setlength{\baselineskip}{0in}\rule[-0.5mm]{\textwidth}{0.5pt}\par}%
  {\hspace{\stretch{3}}%
    web: \web\hspace{\stretch{1}}%
    email: \email\hspace{\stretch{1}}%
    phone: \texttt{+}\phone\hspace{0.5em}%
  }%
  \vspace{0.5em}%
}

% 1 2  <-Use this to define your schools, companies, and organisations.
\newenvironment{entity}[2]{%
    \textbf{#1}\hfill #2\par
}{
    \vspace{0.1em}%
    \par%
}
\newenvironment{position}[2]{%
    \ifthenelse{\not\equal{#1}{}\or\not\equal{#2}{}}{\textsl{#1}\hfill#2\par}
}{
    \vspace{0.3em}%
}

\begin{document}
\makeheader
\section{Objective}

An entry-level position in systems administration or software development.
\section{Employment}

    \begin{entity}{\href{http://hostgator.com}{HostGator.Com, LLC}}{Austin, TX}

        \begin{position}{Systems Administrator}{June 2011 to July 2012}

-- Resolved hundreds of Linux-related issues per week through a support ticket system, including:\par
  - PHP/MySQL/Rails/website/application errors and issues\par
  - Backups and restorations, using customer provided tarballs or rsync from NAS backup appliances\par
  - Reboot triage and server tuning (particularly MySQL/Percona and Apache)\par
  - Software upgrades and installations (Apache/PHP modules, Redmine, phpBB, etc.)\par
-- Filed internal bug reports and wrote documentation for HG's wikis and knowledgebase.

        \end{position}
        \begin{position}{Systems Monitoring}{December 2011 to July 2012}

-- Monitored $>$4500 servers, using Zabbix (and ZMonitor), in a small team to maintain quality service.\par
-- A typical week consisted of:\par
  - Reducing abnormal load and user abuse (rogue scripts, spam, hacked accounts)\par
  - Checking and reporting hardware health (disk usage, filesystem status, drive swaps, RAID health)\par
  - Restoring network connectivity and deflecting attacks (UDP/SYN floods, Slowloris)\par
  - And correcting broken configurations (network, services).\par
-- Wrote \href{http://forums.hostgator.com/search.php?do=finduser&u=126179}{public announcements} on HG's \href{http://forums.hostgator.com/network-status-f14.html}{network status forums} for extended downtime issues.\par
-- Assisted a Level 3 administrator in migrating the farm from 32bit to 64bit.

        \end{position}

    \end{entity}
\section{Volunteering}

    \begin{entity}{\href{http://commiesubs.com}{Commie Subs}}{Remote}

        \begin{position}{Developer}{June 2012 -- present}

-- Wrote a showtime and volunteer tracking application using the \href{http://slimframework.com}{Slim Framework}.\par
-- I write scripts and applications that let volunteers be a little more lazy.

        \end{position}

    \end{entity}
    \begin{entity}{\href{http://knightsofreason.net}{Knights of Reason}}{Remote}

        \begin{position}{Moderator/System Administrator}{March 2008 -- present}

-- I moderate(d) the KoRx game servers and KoR community and organize events.\par
-- I co-manage and maintain documentation for the server and am the primary contact for technical issues.

        \end{position}

    \end{entity}
    \begin{entity}{\href{http://thebikeproject.org}{The Bike Project of Urbana-Champaign}}{Urbana, IL}

        \begin{position}{Bike Mechanic/Technologist}{August 2010 to February 2011}

-- Repurposed old laptops with Debian for use at the shops, and managed membership data.

        \end{position}

    \end{entity}
\section{Projects}

    \begin{entity}{\href{https://github.com/liliff/zmonitor}{ZMonitor}}{}

Terminal dashboard developed in Ruby for the Zabbix monitoring suite. Configured in YAML, it uses JSON to retrieve active alerts and answer them with the Zabbix API. Known for mass pattern-based (using regex) acknowledgement (useful during network failures) and alternative plaintext output for piping or copying.

    \end{entity}
\section{Skills}

\textbf{OS/Distros:} CentOS, Arch Linux, Gentoo Linux, Debian, Windows XP\par
\textbf{Code:} Bash, PHP, Ruby, Javascript\par
\textbf{Markup:} Markdown, HTML5, CSS3, YAML, \LaTeX, JSON\par
\textbf{HTTP:} Lighttpd, Apache, nginx\par
\textbf{Database:} MySQL, Redis\par
\textbf{Monitoring:} Zabbix, sysstat, IPMI, tcpdump/ngrep, iptables\par
\textbf{Package Managers:} portage, pacman, yum/rpm, apt\par
\textbf{Miscellaneous:} cPanel/WHM, Git, LAMP, Compilation, Jekyll
\section{Education}

    \begin{entity}{University of Illinois at Urbana-Champaign}{2010 -- 2011}


    \end{entity}
\section{Links}

\textbf{Github:} \url{www.github.com/liliff}\par
\textbf{LinkedIn:} \url{www.linkedin.com/in/musee}
\\
\\
\textsl{Source:} \url{https://raw.github.com/liliff/resume/master/resume.md}
\end{document}
